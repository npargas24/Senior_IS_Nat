%!TEX root = ../main.tex\
\chapter{Introduction}\label{intro}

\section{What is Reverse Engineering?}
Reverse Engineering is the process of disassembling something in order to understand how it works. 
It has been done long before it was used for technology. For example, the dissections that were a common high school experience.
They were done to break down something as mystifying as life into clear components and functionalities. 
One of the best ways to fully understand how something works is to open it up.

Reverse engineering software can be done in many ways. 
The methods of analysis can be divided into two categories, static and dynamic analysis. 
Static analysis involves combing over a snapshot of the code in a single state.
 Dynamic analysis is done when the program is being executed and changing states. To start reverse engineering, usually a program’s executable is passed to a disassembler which will break it down into the assembly level code. 
A programmer can use system monitoring tools to display information that was gathered by the operating system on the program and how it interacts with its environment. 
A debugger shows what happens in the CPU with the disassembled code one line at a time.
Using these different tools, the programmer must use their intuition to guide the reversing process. 

There are many reasons why someone might want to reverse engineer software. 
The only time that the workings of software is available to anyone is when that software is open source.
Open source software is freely available source code that anyone can view, share, and modify. 
However, software developers sometimes opt out of publicly distributing the source code. 
There are a number of potential causes for this, including licensing and ownership issues, the need to preserve proprietary data, or other personal reasons.

Some of the most common reasons a person would choose to reverse engineer software is so that they can perform malware analysis, improve programming skills, recover lost source code, and implement interoperability between programs. 
My first introduction to reverse engineering was through a video where the creator was reverse engineering Apple’s Facetime for Mac so he could reinstate closed captioning for his hearing impaired mother. 
The list of uses is as extensive as programs out there.

 Most software developers know reverse engineering in the context of malware analysis. 
 A classic example is a Trojan virus. Malicious developers can hide malware in programs that are designed to look innocuous. 
 Breaking apart these programs will unveil the malware hidden inside. Developers also often call on external libraries. 
 With large projects, keeping track of dependencies can be hard. Hackers can hide entry points in these libraries so they can access unauthorized information. 
 Reverse engineering is the only foolproof way to reveal exactly what is happening at each step of a program’s execution.

Learning how something is being done can be great to improve skills. 
Reverse engineering requires extensive knowledge of both assembly code and how high level code structuring. 
Reverse engineering someone else's code is no simple feat and will cause the reverser to need to learn in depth the possible ways a program could be constructed. 
This project will be an endeavor in learning both forward engineering and reverse engineering.  

People may reverse engineer their own program to recover lost source code. 
While this is no easy task, it is possible to use reverse engineering to recover source code. 
Many software developers know the pain of losing source code and, with an executable, reverse engineering offers a potential solution.

Interoperability is one of the reasons a person might choose to reverse engineer something. 
This purpose is also only able to be accomplished through reverse engineering. 
When working with external libraries or operating systems that have documentation on usage but no source code, oftentimes documentation doesn’t cover all use cases. 
While a programmer could keep throwing things at the wall or try to contact the vendor, reverse engineering provides a surefire way of figuring things out.

\section{Java RE}
\subsection{JVM}
\subsection{Reversing Bytecode}
\subsection{Java Name Obfuscation}
\subsubsection{Obfuscating names for .class file}

\section{Mobile RE}
\subsection{Hybrid Approach}
\subsection{Symbolic Execution}
\subsection{Generating UML Diagram}
\subsection{Android}
\subsubsection{Tools}
\paragraph{Dalvik VM Android}
\paragraph{APK Tools}
\subsection{Apple?}
\subsubsection{Obj C tools}
